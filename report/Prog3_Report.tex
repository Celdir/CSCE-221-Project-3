%% LyX 2.1.4 created this file.  For more info, see http://www.lyx.org/.
%% Do not edit unless you really know what you are doing.
\documentclass{article}
\usepackage[latin9]{inputenc}
\usepackage[top=0.6in,left=0.6in,right=0.6in,bottom=0.6in]{geometry}
\usepackage{pgfplots}
\usepackage[final]{pdfpages}
\pgfplotsset{width=16cm,compat=1.9}
\setlength{\parindent}{0cm}

\usepackage{amsmath}
\usepackage{units}

\begin{document}

\title{\textbf{Programming 2 Assignment}}


\author{Michael Earl, Gerald Hu \\
 \\
 CSCE 221-200}


\date{April 11, 2016}

\maketitle

\section*{Introduction}

\noindent The purpose of this assignment was to analyze the theoretical
and actual performance of several common sorting algorithms discussed
in class. The project implemented bubble sort, two ``slow sorting''
algorithms (insertion and selection sort), and two ``fast sorting''
algorithms (mergesort and quicksort). The running time of these algorithms
was analyzed for several types of data: sorted data, reverse-sorted
data, randomized data, and data with many similar elements. 
Insertion Sort and Bubble Sort ran quickly on already-sorted
data, but performed worse in other areas. Out of the slow sorts, Insertion
was the best, and Bubble was the worst in most cases. Out of all sorting
algorithms, Quicksort and Mergesort performed similarly, but Quicksort
was a bit faster. Selection sort ran in $\mathcal{O}(n^2)$ in all cases,
insertion sort and bubble sort ran in $\mathcal{O}(n^2)$ in the average
and worst cases, but in $\mathcal{O}(n)$ in the best case. Mergesort runs
in $\mathcal{O}(n\log n)$ in all cases, and quicksort runs in $\mathcal{O}(n\log n)$ 
in the best and average cases, but in $\mathcal{O}(n^2)$ in the worst case. 
We learned that some algorithms perform better in certain situations than others,
namely that insertion sort is very good at sorting already sorted and almost sorted
sequences, but quicksort is better for just about everything else. \\


\section*{Implementation Details}

\noindent All sorting algorithms took random-access iterators to the
first and last elements of the container, and a comparator. \\


\noindent Bubble Sort is a simple sorting algorithm. It compares a
set of two elements that are next to each other, and if the elements
in the bubble are out of order, they are swapped. This set traverses
the entire container from start to end, then starts from the beginning
again, repeating until there are no more swaps to make.\\


\noindent Selection Sort is one of the two ``slow'' sorting algorithms.
It traverses the entire container searching for the smallest element,
then swaps that element into the first position. Then it searches
the remainder of the container for the next-smallest element, then
swaps that element into the first position of the remainder. On its
$i$th traversal, it will swap the smallest element into the $i$th
position. This process continues $n$ times, at which point there
is no more ''remainder'' of the container to traverse.\\


\noindent Insertion Sort is the other of the two ``slow'' sorting
algorithms. It works by keeping an already-sorted area at the beginning
of the container. And swapping the current element back through the already-sorted
area until it reaches the correct spot. In each
traversal, for each step $i$, if $i$ is not the first element, and
if $i$ and $i-1$ are out of order, the two are swapped, swapping
the element into an ''already-sorted'' portion of the container.
Each traversal will grow the already-sorted portion, until all the
data is in the sorted portion. \\


\noindent Mergesort is one of the two ``fast'' sorting algorithms.
It splits the input data into two subarrays, and recursively splits
each of those subarrays into two subarrays, repeating the splitting
process until each subarray is of size$<$2. These subarrays are then
sorted in sets of two, and the resulting sorted data is put into an
array and returned. The returned sorted data is merged with more data
using the same merge function, until fully-sorted data is returned.\\


\noindent Quicksort is the other of the two ``fast'' sorting algorithms.
It randomly selects an element of the data to use as a ``pivot''
, and traverses the entire container. It moves all elements less than
the pivot to before the pivot's position, and moves all elements greater
than the pivot to after the pivot's position, all of which is done
in-place. This places the pivot in the proper position. Then, quicksort
is recursively called, once on the set of elements less than the pivot,
and once on the set of elements greater than the pivot. \\


\noindent The skeleton of timing.cpp was provided by the instructors.
The timing function was already implemented for randomly sorted input
data; we modified it to analyze sorted and reverse-sorted data, as
well as data with many similar elements. The timing function relies
on high\_resolution\_clock; given a sorting function and a number
$n$, the timing function would generate a vector with $n$ elements
in it. Then, the timing function would try to sort the vector using
the specified sorting algorithm. This process was repeated for increasing
sizes of $n$, and repeated 10 times at each $n$ to ensure an accurate
average time. {[}TODO - what we learned{]}\\



\section*{Theoretical Analysis}

\noindent Bubble Sort is $O(n^{2})$ average and worst case. It will
force the largest element to the end after $n$ comparisons; then
it will repeat this for the $n-1$ smallest element, then the $n-2$
smallest element, and so on, repeating $n$ times in total. $n$ traversals
and $n$ comparisons/swaps at each traversal results in $O(n^{2})$
time. Bubble Sort's best case is $O(n)$, in the case that it's already
sorted. It traverses the list once and makes $n$ comparisons, but
does not do any swaps or any further traversals.\\


\noindent Selection Sort is $O(n^{2})$ in all cases. In searching
for the smallest element, it will traverse $n$ elements and make
$n$ comparisons, before forcing the smallest element to the start.
It then repeats this process for a sublist of size $n-1$, then $n-2$...$1$.
$\sum_{i=0}^{n}i=(i)(i+1)/2$, which is $O(n^{2})$ behavior. Furthermore,
it will always search for the smallest element, regardless of whether
it needs to be swapped or not, and will perform all traversals regardless
of what data type it's given.\\


\noindent Insertion Sort is $O(n^{2})$ average and worst case, for
similar reasons as Selection Sort: it will traverse a list of size
$1$, then size $2$, then $3$...$n$ in the worst case. Unlike Selection Sort,
Insertion Sort is $O(n)$, in the best case that it is already sorted,
as it will only traverse the list once. \\


\noindent Mergesort is $O(nlog(n))$ in all cases. At each step, the
problem is split in half, until it reaches the smallest possible subproblem;
it will take $O(log n)$ splits to reach the base level, and $O(n)$
to merge all the data back together at each level. There is not a ``best case''
input type, as mergesort will perform the same operations regardless
of the data it's given. \\


\noindent Quicksort is $O(nlog(n))$ average case and best case, and
$O(n^{2})$ in a rare worst case. The best case of quicksort is $O(nlogn)$
due to an inherent limit on comparison-based sorting algorithms. Assuming
an ``average case'' where the pivots selected give equal (or close
to equal) sized sublists, it will take $O(n)$ to parition the sequence around the pivot
at every level and $O(logn)$ splits to reach the base level. The
worst case is if every pivot chosen is a bad selection, splitting
into a subproblem of size 1 and size $(n-1)$, which is then split
into a subproblem of size 1 and size $(n-2)$, so on....resulting
in $O(n)$ splits, which means $O(n^2)$ time in the worst case. \\



\section*{Experimental Setup}

Timing tests were conducted using the provided timing.cpp, compiled
with the provided makefile's commands. Compilation was done on Michael's
desktop, with G++ version 5.3.0. Compilation was set to the C++14
standard, with the -G flag enabled and O2 optimization level, warnings
set to -Wall -Werror (warn all and all warnings treated as compilation
errors), and dependencies flagged with -MMD (auto-generate dependencies).\\


\noindent Tests were run on Michael's desktop, which runs Arch Linux
x86\_64 version 4.4.5-1. It has 8 total gigabytes of RAM. It uses
an AMD FX-8350 8-Core processor, with a clock speed of 4 GHz. \\


\noindent Timing functions output timing results for input sizes that
were powers of 2, starting from 2 itself, and ending at a maximum
size specified by the user. Each step of the timing was repeated 10
times, and the average of each result taken. Bubble Sort went up to
a maximum input size of 32768 $(2^{15})$ elements; Insertion and
Selection sorts went up to a maximum input size of 131072 $(2^{17})$
elements; and Quicksort and Mergesort went up to a maximum input size
of 4194304 $(2^{22})$ elements.\\


\noindent Four data types were used in testing: sorted data, reverse-sorted
data, randomly-sorted data, and data with ``few unique elements''.
For our purposes, ``data with few unique elements'' was assumed
to mean ``data where a lot of the elements are the same''; the input
data generated consisted mostly of 1s, with a few 2s spaced out in
the data.\\



\section*{Results and Discussion}

\begin{tikzpicture}
\begin{axis}[
    title={Bubble sort on several sequence types},
    xlabel={Input size},
    ylabel={Time},
    xmode = log,
    log basis x=2,
    ymode = log,
    log basis y=10,
    legend style={legend pos=north west},
    cycle list name=black white
]
\addplot table[x expr={\thisrowno{0}}, y expr={\thisrowno{1}}] {../timing/bubble_random.dat};
\addplot table[x expr={\thisrowno{0}}, y expr={\thisrowno{1}}] {../timing/bubble_pre_sorted.dat};
\addplot table[x expr={\thisrowno{0}}, y expr={\thisrowno{1}}] {../timing/bubble_reverse.dat};
\addplot table[x expr={\thisrowno{0}}, y expr={\thisrowno{1}}] {../timing/bubble_few_unique.dat};
\legend{Random Sequence, Pre-Sorted Sequence, Reverse Sorted Sequence, Sequence with Few Unique Elements};
\end{axis}
\end{tikzpicture}

Bubble sort performs very similarly on random sequences, reverse sorted sequences, and sequences with few unique elements. It performs nearly \textit{identically} on reverse sorted sequences and sequences with few unique elements. The plots of all three of these seem to follow a quadratic curve, and thus appear to run in $\mathcal{O}(n^2)$ time. The random sequence seems to take slightly longer, most likely because of the bit of extra time it takes to generate the random numbers. On pre-sorted sequences, bubble sort runs very quickly, much more quickly than the other three. The plot for this is clearly linear, which is because bubble sort should only iterate through the sequence once before realizing that it is sorted. \\

\begin{tikzpicture}
\begin{axis}[
    title={Selection sort on several sequence types},
    xlabel={Input size},
    ylabel={Time},
    xmode = log,
    log basis x=2,
    ymode = log,
    log basis y=10,
    legend style={legend pos=north west},
    cycle list name=black white
]
\addplot table[x expr={\thisrowno{0}}, y expr={\thisrowno{1}}] {../timing/selection_random.dat};
\addplot table[x expr={\thisrowno{0}}, y expr={\thisrowno{1}}] {../timing/selection_pre_sorted.dat};
\addplot table[x expr={\thisrowno{0}}, y expr={\thisrowno{1}}] {../timing/selection_reverse.dat};
\addplot table[x expr={\thisrowno{0}}, y expr={\thisrowno{1}}] {../timing/selection_few_unique.dat};
\legend{Random Sequence, Pre-Sorted Sequence, Reverse Sorted Sequence, Sequence with Few Unique Elements};
\end{axis}
\end{tikzpicture}

Selection sort performs virtually identically on every type of sequence. This is because no matter the sequence, selection sort will iterate through $n-1$ elements, then $n-2$ elements, and so on, finding the smallest element each time and putting it in the correct position. Its curve is quadratic, so it runs in $\mathcal{O}(n^2)$. \\

\begin{tikzpicture}
\begin{axis}[
    title={Insertion sort on several sequence types},
    xlabel={Input size},
    ylabel={Time},
    xmode = log,
    log basis x=2,
    ymode = log,
    log basis y=10,
    legend style={legend pos=north west},
    cycle list name=black white
]
\addplot table[x expr={\thisrowno{0}}, y expr={\thisrowno{1}}] {../timing/insertion_random.dat};
\addplot table[x expr={\thisrowno{0}}, y expr={\thisrowno{1}}] {../timing/insertion_pre_sorted.dat};
\addplot table[x expr={\thisrowno{0}}, y expr={\thisrowno{1}}] {../timing/insertion_reverse.dat};
\addplot table[x expr={\thisrowno{0}}, y expr={\thisrowno{1}}] {../timing/insertion_few_unique.dat};
\legend{Random Sequence, Pre-Sorted Sequence, Reverse Sorted Sequence, Sequence with Few Unique Elements};
\end{axis}
\end{tikzpicture}

Insertion sort performs fastest on pre-sorted sequences, quite predictably. It simply iterates through the sequence once and doesn't swap anything. The plot is predictably linear, implying $\mathcal{O}(n)$ time complexity on pre-sorted sequences. The other three sequence types all seem to have quadratic curves, implying that insertion sort runs in $\mathcal{O}(n^2)$ on those sequences. These three sequences do have different coefficients, though. Insertion sort is slowest on reverse sorted sequences because it is guaranteed to run in $n-1 + n-2 + ... + 1$ operations, which is identical to selection sort. It is slightly faster on random sequences, which are not guaranteed absolutely worst case performance. It runs a bit faster on sequences with few unique elements than it does on random sequences, likely because it doesn't have to make quite as many swaps because so many elements are identical. \\

\begin{tikzpicture}
\begin{axis}[
    title={Mergesort on several sequence types},
    xlabel={Input size},
    ylabel={Time},
    xmode = log,
    log basis x=2,
    ymode = log,
    log basis y=10,
    legend style={legend pos=north west},
    cycle list name=black white
]
\addplot table[x expr={\thisrowno{0}}, y expr={\thisrowno{1}}] {../timing/merge_random.dat};
\addplot table[x expr={\thisrowno{0}}, y expr={\thisrowno{1}}] {../timing/merge_pre_sorted.dat};
\addplot table[x expr={\thisrowno{0}}, y expr={\thisrowno{1}}] {../timing/merge_reverse.dat};
\addplot table[x expr={\thisrowno{0}}, y expr={\thisrowno{1}}] {../timing/merge_few_unique.dat};
\legend{Random Sequence, Pre-Sorted Sequence, Reverse Sorted Sequence, Sequence with Few Unique Elements};
\end{axis}
\end{tikzpicture}

Mergesort runs very similarly on all four sequence types. It is essentially identical for pre-sorted sequences, reverse sorted sequences, and sequences with few unique elements, and slightly slower on random sequences because of the small overhead of generating random numbers. Mergesort performs so similarly on different sequence types is because it does the same steps each time: recursively splitting the sequence in half and then merging the halves back together. The curves \textit{appear} to be linear with a high coefficient, but they actually curve up very slightly because mergesort always runs in $\mathcal{O}(n\log n)$. \\

\begin{tikzpicture}
\begin{axis}[
    title={Quicksort on several sequence types},
    xlabel={Input size},
    ylabel={Time},
    xmode = log,
    log basis x=2,
    ymode = log,
    log basis y=10,
    legend style={legend pos=north west},
    cycle list name=black white
]
\addplot table[x expr={\thisrowno{0}}, y expr={\thisrowno{1}}] {../timing/quick_random.dat};
\addplot table[x expr={\thisrowno{0}}, y expr={\thisrowno{1}}] {../timing/quick_pre_sorted.dat};
\addplot table[x expr={\thisrowno{0}}, y expr={\thisrowno{1}}] {../timing/quick_reverse.dat};
\addplot table[x expr={\thisrowno{0}}, y expr={\thisrowno{1}}] {../timing/quick_few_unique.dat};
\legend{Random Sequence, Pre-Sorted Sequence, Reverse Sorted Sequence, Sequence with Few Unique Elements};
\end{axis}
\end{tikzpicture}

Quicksort runs very similarly on all four sequence types. It is essentially identical for pre-sorted sequences, reverse sorted sequences, and sequences with few unique elements, and slightly slower on random sequences because of the small overhead of generating random numbers. The curves \textit{appear} to be linear with a high coefficient, but they actually curve up very slightly because quicksort almost always runs in $\mathcal{O}(n\log n)$. \\

\begin{tikzpicture}
\begin{axis}[
    title={Comparison of sorts on random sequences},
    xlabel={Input size},
    ylabel={Time},
    xmode = log,
    log basis x=2,
    ymode = log,
    log basis y=10,
    legend style={legend pos=north west},
    cycle list name=black white
]
\addplot table[x expr={\thisrowno{0}}, y expr={\thisrowno{1}}] {../timing/bubble_random.dat};
\addplot table[x expr={\thisrowno{0}}, y expr={\thisrowno{1}}] {../timing/selection_random.dat};
\addplot table[x expr={\thisrowno{0}}, y expr={\thisrowno{1}}] {../timing/insertion_random.dat};
\addplot table[x expr={\thisrowno{0}}, y expr={\thisrowno{1}}] {../timing/merge_random.dat};
\addplot table[x expr={\thisrowno{0}}, y expr={\thisrowno{1}}] {../timing/quick_random.dat};
\legend{Bubble sort, Selection sort, Insertion sort, Mergesort, Quicksort};
\end{axis}
\end{tikzpicture}

Quicksort and mergesort are clearly the fastest on random sequences (they are both $\mathcal{O}(n\log n)$), although quicksort has slightly better coefficients. Insertion, selection, and bubble sort all take quadratic time (as indicated by their quadratic curves), but insertion sort has the best coefficents and bubble sort has the worst coefficients. \\

\begin{tikzpicture}
\begin{axis}[
    title={Comparison of sorts on pre-sorted sequences},
    xlabel={Input size},
    ylabel={Time},
    xmode = log,
    log basis x=2,
    ymode = log,
    log basis y=10,
    legend style={legend pos=north west},
    cycle list name=black white
]
\addplot table[x expr={\thisrowno{0}}, y expr={\thisrowno{1}}] {../timing/bubble_pre_sorted.dat};
\addplot table[x expr={\thisrowno{0}}, y expr={\thisrowno{1}}] {../timing/selection_pre_sorted.dat};
\addplot table[x expr={\thisrowno{0}}, y expr={\thisrowno{1}}] {../timing/insertion_pre_sorted.dat};
\addplot table[x expr={\thisrowno{0}}, y expr={\thisrowno{1}}] {../timing/merge_pre_sorted.dat};
\addplot table[x expr={\thisrowno{0}}, y expr={\thisrowno{1}}] {../timing/quick_pre_sorted.dat};
\legend{Bubble sort, Selection sort, Insertion sort, Mergesort, Quicksort};
\end{axis}
\end{tikzpicture}

On pre-sorted sequences, insertion sort and bubble sort are the fastest, and their plots are nearly identical. This is because they both simply iterate through the sequence once and stop. Their plots are linear with some mild inconsistencies which can be attributed to changing coefficients, probably due to memory allocation overhead. Quicksort and mergesort are once again very similar, with quicksort being slightly faster again. They don't run any faster than they did on randomized sequences because of their divide and conquer strategies. They are, as usual, $\mathcal{O}(n\log n)$. Selection sort is by far the slowest in this case. While the other slower algorithms performed in linear time on presorted sequences, selection sort did not because it always runs the same way: iterating through the entire array after its current position to find the smallest element and swap it to the right place. Selection sort is, as always, quadratic. \\

\begin{tikzpicture}
\begin{axis}[
    title={Comparison of sorts on reverse sorted sequences},
    xlabel={Input size},
    ylabel={Time},
    xmode = log,
    log basis x=2,
    ymode = log,
    log basis y=10,
    legend style={legend pos=north west},
    cycle list name=black white
]
\addplot table[x expr={\thisrowno{0}}, y expr={\thisrowno{1}}] {../timing/bubble_reverse.dat};
\addplot table[x expr={\thisrowno{0}}, y expr={\thisrowno{1}}] {../timing/selection_reverse.dat};
\addplot table[x expr={\thisrowno{0}}, y expr={\thisrowno{1}}] {../timing/insertion_reverse.dat};
\addplot table[x expr={\thisrowno{0}}, y expr={\thisrowno{1}}] {../timing/merge_reverse.dat};
\addplot table[x expr={\thisrowno{0}}, y expr={\thisrowno{1}}] {../timing/quick_reverse.dat};
\legend{Bubble sort, Selection sort, Insertion sort, Mergesort, Quicksort};
\end{axis}
\end{tikzpicture}

On reverse sorted sequences, mergesort and quicksort perform just like they have on our other sequence types: in $\mathcal{O}(n\log n)$ time with quicksort having better coefficients. Selection and insertion sort perform identically on this sequence type (both quadratically) because insertion sort will have to go through the same number of steps that selection sort does, but in reverse: $1 + 2 + ... + n-1$ as opposed to selection sort's $n-1 + n-2 + ... + 1$. This occurs because insertion sort will have to move each successive element all the way back through the sequence. Bubble sort is similar, but with worse coefficients. \\

\begin{tikzpicture}
\begin{axis}[
    title={Comparison of sorts on sequences with few unique elements},
    xlabel={Input size},
    ylabel={Time},
    xmode = log,
    log basis x=2,
    ymode = log,
    log basis y=10,
    legend style={legend pos=north west},
    cycle list name=black white
]
\addplot table[x expr={\thisrowno{0}}, y expr={\thisrowno{1}}] {../timing/bubble_few_unique.dat};
\addplot table[x expr={\thisrowno{0}}, y expr={\thisrowno{1}}] {../timing/selection_few_unique.dat};
\addplot table[x expr={\thisrowno{0}}, y expr={\thisrowno{1}}] {../timing/insertion_few_unique.dat};
\addplot table[x expr={\thisrowno{0}}, y expr={\thisrowno{1}}] {../timing/merge_few_unique.dat};
\addplot table[x expr={\thisrowno{0}}, y expr={\thisrowno{1}}] {../timing/quick_few_unique.dat};
\legend{Bubble sort, Selection sort, Insertion sort, Mergesort, Quicksort};
\end{axis}
\end{tikzpicture}

On sequences with few unique elements, quicksort and mergesort perform as they have with every other sequence type. Insertion sort runs faster than quicksort on sequences with fewer than $2^9$ elements. This is likely because insertion has to perform relatively few swaps because large portions of the sequence will already be sorted because there are large patches of identical elements. Eventually, however, its quadratic performance outweighs its good coefficients (which are due to this unique sequence type) and it becomes slower than both quicksort and mergesort. Selection and bubble sort remain considerably slower, however, due to worse coefficients. They don't get any benefits from this sequence type. \\

\begin{tikzpicture}
\begin{axis}[
    title={Constants of sorts on random sequences},
    xlabel={Input size},
    ylabel={Time / Expected Time},
    xmode = log,
    log basis x=2,
    ymode = log,
    log basis y=10,
    legend style={legend pos=north west},
    cycle list name=black white
]
\addplot table[x expr={\thisrowno{0}}, y expr={\thisrowno{1} / (\thisrowno{0} * \thisrowno{0})}] {../timing/bubble_random.dat};
\addplot table[x expr={\thisrowno{0}}, y expr={\thisrowno{1} / (\thisrowno{0} * \thisrowno{0})}] {../timing/selection_random.dat};
\addplot table[x expr={\thisrowno{0}}, y expr={\thisrowno{1} / (\thisrowno{0} * \thisrowno{0})}] {../timing/insertion_random.dat};
\addplot table[x expr={\thisrowno{0}}, y expr={\thisrowno{1} / (\thisrowno{0} * \thisrowno{2})}] {../timing/merge_random.dat};
\addplot table[x expr={\thisrowno{0}}, y expr={\thisrowno{1} / (\thisrowno{0} * \thisrowno{2})}] {../timing/quick_random.dat};
\legend{Bubble sort, Selection sort, Insertion sort, Mergesort, Quicksort};
\end{axis}
\end{tikzpicture}

The constants of all of these algorithms improve at first and then level out after a certain point. We hypothesize that this is due to runtime optimizations. The effect is more dramatic with the slower algorithms than it is with the faster algorithms, although we are not sure why. The coefficients of the slower sorts aren't really comparable with the coefficients of the quicker sorts because they are asymptotically different ($\mathcal{O}(n^2)$ vs $\mathcal{O}(n\log n)$), so we will compare them separately. For the slow sorts, insertion sort has the best coefficients and bubble sort has the worst, while selection sort is in the middle. Insertion sort has the best coefficients because it often has to traverse smaller portions of the sequence than selection sort and bubble sort. Selection sort is faster than bubble sort because it performs far fewer swaps than bubble does. In fact, it performs at most $n$ swaps, while bubble sort performs at most $\mathcal{O}(n^2)$ swaps. For the fast sorts, quicksort has a better coefficient than mergesort does. We suspect that this is largely due to mergesort's extra overhead for allocating extra memory, which makes the merge step slower than quicksort's pivoting technique. \\



\section*{Conclusion}

This assignment showed the differences in runtime between various
sorting algorithms, highlighting which ones were theoretically more
efficient, and which ones actually performed better in practice. Quicksort,
like the discussions in class, consistently performed well and sorted
quickly; Insertion Sort and Bubble Sort ran quickly on already-sorted
data, but performed worse in other areas. Out of the slow sorts, Insertion
was the best, and Bubble was the worst in most cases. Out of all sorting
algorithms, Quicksort and Mergesort performed similarly, but Quicksort
was a bit faster. 
\end{document}
